\documentclass[xcolor=x11names,compress]{beamer}

%% General document %%%%%%%%%%%%%%%%%%%%%%%%%%%%%%%%%%
\usepackage{graphicx}
\usepackage{tikz}
\usepackage{Tabbing}
\usetikzlibrary{decorations.fractals}
\usepackage{fancyvrb}
%%%%%%%%%%%%%%%%%%%%%%%%%%%%%%%%%%%%%%%%%%%%%%%%%%%%%%

%% Beamer Layout %%%%%%%%%%%%%%%%%%%%%%%%%%%%%%%%%%
\useoutertheme[subsection=false,shadow]{miniframes}
\useinnertheme{default}
\usefonttheme{serif}
\usepackage{palatino}
\usepackage{tabu}
% Links
\usepackage{hyperref}
\definecolor{links}{HTML}{003262}
\hypersetup{colorlinks,linkcolor=,urlcolor=links}

% addition of color
\usepackage{xcolor}
\definecolor{CoolBlack}{rgb}{0.0, 0.18, 0.39}
\definecolor{byellow}{rgb}{0.55037, 0.38821, 0.06142}
\definecolor{dgreen}{rgb}{0.,0.6,0.}
\definecolor{RawSienna}{cmyk}{0,0.72,1,0.45}
\definecolor{forestgreen(web)}{rgb}{0.13, 0.55, 0.13}
\definecolor{cardinal}{rgb}{0.77, 0.12, 0.23}

\setbeamerfont{title like}{shape=\scshape}
\setbeamerfont{frametitle}{shape=\scshape}

\setbeamercolor*{lower separation line head}{bg=CoolBlack} 
\setbeamercolor*{normal text}{fg=black,bg=white} 
\setbeamercolor*{alerted text}{fg=byellow} % just testing; I think this looks better
\setbeamercolor*{example text}{fg=black} 
\setbeamercolor*{structure}{fg=black} 
 
\setbeamercolor*{palette tertiary}{fg=black,bg=black!10} 
\setbeamercolor*{palette quaternary}{fg=black,bg=black!10} 

% Margins
\usepackage{changepage}

\mode<presentation>
{
  \definecolor{berkeleyblue}{HTML}{003262}
  \definecolor{berkeleygold}{HTML}{FDB515}
  \usetheme{Boadilla}      % or try Darmstadt, Madrid, Warsaw, Boadilla...
  %\usecolortheme{dove} % or try albatross, beaver, crane, ...
  \setbeamercolor{structure}{fg=berkeleyblue,bg=berkeleygold}
  \setbeamercolor{palette primary}{bg=berkeleyblue,fg=white} % changed this
  \setbeamercolor{palette secondary}{fg=berkeleyblue,bg=berkeleygold} % changed this
  \setbeamercolor{palette tertiary}{bg=berkeleyblue,fg=white} % changed this
  \usefonttheme{structurebold}  % or try serif, structurebold, ...
  \useinnertheme{circles}
  \setbeamertemplate{navigation symbols}{}
  \setbeamertemplate{caption}[numbered]
  \usebackgroundtemplate{}
}

% Columns
\renewcommand{\(}{\begin{columns}}
\renewcommand{\)}{\end{columns}}
\newcommand{\<}[1]{\begin{column}{#1}}
\renewcommand{\>}{\end{column}}

% adding slide numbers
\addtobeamertemplate{navigation symbols}{}{%
    \usebeamerfont{footline}%
    \usebeamercolor[fg]{footline}%
    \hspace{1em}%
    \insertframenumber/\inserttotalframenumber
}

% equation stuff
\newcommand{\Macro}{\ensuremath{\Sigma}}
\newcommand{\Sn}{\ensuremath{S_N} }
\newcommand{\vOmega}{\ensuremath{\hat{\Omega}}}
\usepackage{mathrsfs}
\usepackage[mathcal]{euscript}
\usepackage{amssymb}
\usepackage{amsthm}
\usepackage{epsfig}
\usepackage{amsmath}
\newcommand{\ve}[1]{\ensuremath{\mathbf{#1}}}
\newcommand{\micro}{\ensuremath{\sigma}}
\newcommand{\detR}{\ensuremath{\Sigma}}

% title stuff for footer
\title{The PyNE Software Library}
\author{R.\ N.\ Slaybaugh}
\date{22 October 2014}

%%%%%%%%%%%%%%%%%%%%%%%%%%%%%%%%%%%%%%%%%%%%%%%%%%%%%%
\begin{document}



%%%%%%%%%%%%%%%%%%%%%%%%%%%%%%%%%%%%%%%%%%%%%%%%%%%%%%
%%%%%%%%%%%%%%%%%%%%%%%%%%%%%%%%%%%%%%%%%%%%%%%%%%%%%%
\begin{frame}
\title{The PyNE Software Library: Why and How?}
%\subtitle{}
\author{
        \includegraphics[height=2cm]{../bk-eps-converted-to}\\R.\ N.\ Slaybaugh \\ Univ.\ of Cal.\ Berkeley}

\date{ANS Nor Cal Meeting \\ 22 October 2014\\ Alfred's Stakehouse, San Francisco, CA}
\titlepage
\end{frame}

%------------------------------------------------------
\begin{frame}{Outline}

	\begin{columns}
  	\begin{column}{0.4\textwidth}
	    \begin{itemize}
        \item PyNE \cite{pyne}: what is it?
        \begin{itemize}
            \item Demo
        \end{itemize}
        \item Current initiatives
        \item PyNE as a research tool
        \item Get involved!
	    \end{itemize}
  	\end{column}
 	%
 	\begin{column}{0.5\textwidth}
 	   \begin{center}
 	   \begin{figure}
       \includegraphics[height=3.5cm]{pyne-icon-big}
% 	   \includegraphics[height=1.25in,clip]{data_sources_thumb}  \\
%       \includegraphics[height=1.25in,clip]{half_life_thumb}
	   \end{figure}
 	   \end{center}
  	\end{column}
	\end{columns}

\end{frame}

%%%%%%%%%%%%%%%%%%%%%%%%%%%%%%%%%%%%%%%%%%%%%%%%%%%%%%
%%%%%%%%%%%%%%%%%%%%%%%%%%%%%%%%%%%%%%%%%%%%%%%%%%%%%%
\section{PyNE \cite{pyne}: what is it?}
\begin{frame}{What is PyNE?}

    PyNE is \textit{the} open source nuclear engineering toolkit.
    \vspace*{1em}
    \begin{itemize}
    \item PyNE is a \textit{library of composable tools} used to build 
    nuclear science and engineering applications
    \item It is \textit{permissively licensed} (2-clause BSD)
    \item It supports both a \textit{C++} and a \textit{Python} API
    \item The name `PyNE' is a bit of a misnomer since most of the code 
    base is in C++ but most daily usage happens in Python
    \item \textit{v0.4} is the current, stable release
    \item As an organization, PyNE was born in April 2011 
    (however, core parts of PyNE have existed since 2007)
    \end{itemize}

\end{frame}

%------------------------------------------------------
\begin{frame}{What are the Goals of PyNE?}

    \begin{columns}
    \begin{column}{0.4\textwidth}
        To help nuclear engineers:
        \begin{itemize}
        \item be more \alert{productive} (don't reinvent the wheel!)
        \item have the \alert{best solvers}
        \item have a \alert{clear and useful API}
        \item write really \alert{great code}
        \item \alert{teach} the next generation
        \end{itemize}
  	\end{column}
   	%
 	\begin{column}{0.5\textwidth}
 	   \begin{center}
 	   \begin{figure}
 	   \includegraphics[height=1.25in,clip]{data_sources_thumb}  \\
       \includegraphics[height=1.25in,clip]{half_life_thumb}
	   \end{figure}
 	   \end{center}
  	\end{column}
	\end{columns}

\end{frame}

%Python for Nuclear Engineering, or PyNE (http://pyne.io/), is a collaborative, open source project consisting of a collection of
%computational tools pertinent to nuclear engineering analysis and simulations.
%PyNE primarily provides a common Python interface for code written in C++,
%Python, and Fortran. This allows fundamental components of PyNE to easily be
%combined to form powerful and complex programs. These fundamental components
%include canonical nuclide and reaction naming conventions, material handling,
%nuclear data and cross-section reading, mesh operations, and
%physics-code-specific input and output parsing.
%
%This presentation will begin with a discussion about the background and philosophy behind PyNE and include a demonstration of how PyNE could be used in a project. I will also cover some of the current developments, with a focus of how I'm using PyNE as a tool for my research in computational methods for neutral particle transport. 
%%%%%%%%%%%%%%%%%%%%%%%%%%%%%%%%%%%%%%%%%%%%%%%%%%%%%%
%%%%%%%%%%%%%%%%%%%%%%%%%%%%%%%%%%%%%%%%%%%%%%%%%%%%%%
\section{Demo}
\begin{frame}{What Can PyNE Do?}

    \begin{itemize}
    \item Nuclear data and cross-section reading/processing
    \item Material handling
    \item Canonical nuclide and reaction naming conventions
    \item Mesh operations
    \item MCNP and Serpent input/output parsing
    \item Fuel cycle functionality (transmutation, enrichment)
    \item There's more, and the list continues to grow
    \end{itemize}
    
    The idea is to be able to easily combine components and avoid redeveloping
    utilities someone else has developed.

\end{frame}

%------------------------------------------------------
\begin{frame}{Demos}

    \begin{enumerate}
        \item Data sources
        \item Enrichment % TODO maybe add screenshots?
    \end{enumerate}

\end{frame}

%%%%%%%%%%%%%%%%%%%%%%%%%%%%%%%%%%%%%%%%%%%%%%%%%%%%%%
%%%%%%%%%%%%%%%%%%%%%%%%%%%%%%%%%%%%%%%%%%%%%%%%%%%%%%
\section{Current initiatives}
\begin{frame}{What are we Working on Now?}

    The biggest push: orienting toward NQA-1 qualification \cite{pyne_vnv}
    \begin{itemize}
    \item Adding theory descriptions for each piece
    \item Documenting our V\&V process
    \item Slowling removing the ``VnVWarning: pyne.nucname is not yet V\&V compliant'' as
    things become qualified
    \end{itemize}

    \vspace*{1 em}
    DAGMC and R2S development (\textcolor{cardinal}{Univ.\ Wisconsin})
    
    \vspace*{1 em}
    Plug-and-play solver research sandbox (\textcolor{byellow}{Berkeley})
    
\end{frame}

%%%%%%%%%%%%%%%%%%%%%%%%%%%%%%%%%%%%%%%%%%%%%%%%%%%%%%
%%%%%%%%%%%%%%%%%%%%%%%%%%%%%%%%%%%%%%%%%%%%%%%%%%%%%%
\section{PyNE as a research tool}
\begin{frame}{Plug-and-Play Solver Sandbox}

    I study how to better solve the Boltzmann neutral particle transport equation, 
    often with deterministic methods
    \vspace*{1em}
    \begin{itemize}
    \item 
    \end{itemize}

\end{frame}


%%%%%%%%%%%%%%%%%%%%%%%%%%%%%%%%%%%%%%%%%%%%%%%%%%%%%%
%%%%%%%%%%%%%%%%%%%%%%%%%%%%%%%%%%%%%%%%%%%%%%%%%%%%%%
\section{Get involved!}
\begin{frame}{Why Would I Get Involved?}

\begin{block}{As a \alert{user}:}
\begin{itemize}
    \item Do your work or research with PyNE.
    \item Even if you have similar software: When using PyNE, you do not have to develop/maintain everything yourself.
\end{itemize}
\end{block}

\vspace*{1 em}

\begin{block}{As a \alert{developer}:}
\begin{itemize}
    \item \emph{Be selfish!} Contribute to PyNE in ways that support your own work.
    \item PyNE is missing a feature? -- You are probably not the only one who wants it.
    \item Help your future self and others.
\end{itemize}
\end{block}

\end{frame}

%------------------------------------------------------
\begin{frame}{How Can I Get Involved?}

    \begin{block}{Contact PyNE}
    \begin{itemize}
    \item Website: \url{http://pyne.io/}
    \item User's Mailing List: \href{mailto:pyne-users@googlegroups.com}{\nolinkurl{pyne-users@googlegroups.com}}
    \item Developer's List: \href{mailto:pyne-dev@googlegroups.com}{\nolinkurl{pyne-dev@googlegroups.com}}
    \item GitHub: \url{https://github.com/pyne/pyne}
    \end{itemize}
    \end{block}
    
    \vspace*{2 em}

    \begin{block}{What goes into PyNE?}
    Anything that is not export controllable, proprietary, 
    or under HIPPA restrictions!  (If you have questions, \emph{ask})
    \end{block}
  
\end{frame}

% --------------------------------------------------------------
\begin{frame}[fragile]{PyNE In the Literature}

    \begin{itemize}
    \item Intro: ``PyNE: Python For Nuclear Engineering'' \cite{pyne_intro}
    \item Progress reports: \cite{scopatz_pyne}, \cite{pyne_progress}
    \item In research: \cite{Biondo2014}, \cite{MarquezDamian2014280}, \cite{Scopatz2013a}
    \item V\&V: ``Quality Assurance within the PyNE Open Source \\Toolkit'' \cite{pyne_vnv}
    \item Poster at SciPy: \cite{scipy}
    \end{itemize}
  
\end{frame}

%%%%%%%%%%%%%%%%%%%%%%%%%%%%%%%%%%%%%%%%%%%%%%%%%%%%%%
%%%%%%%%%%%%%%%%%%%%%%%%%%%%%%%%%%%%%%%%%%%%%%%%%%%%%%
\section*{}
\begin{frame}[fragile]{Questions?}

    \begin{center}
    \includegraphics[height=3in,clip]{../questions-comic}  
    \end{center}
  
\end{frame}
% --------------------------------------------------------------
\begin{frame}[allowframebreaks]{References}
	\bibliographystyle{unsrt}
	\bibliography{2014-10-norcal-ans-pyne.bib}
\end{frame}

\end{document}
