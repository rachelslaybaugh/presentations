\documentclass[xcolor=x11names,compress]{beamer}

%% General document %%%%%%%%%%%%%%%%%%%%%%%%%%%%%%%%%%
\usepackage{graphicx}
\usepackage{tikz}
\usepackage{Tabbing}
\usetikzlibrary{decorations.fractals}
\usepackage{fancyvrb}
%%%%%%%%%%%%%%%%%%%%%%%%%%%%%%%%%%%%%%%%%%%%%%%%%%%%%%

%% Beamer Layout %%%%%%%%%%%%%%%%%%%%%%%%%%%%%%%%%%
\useoutertheme[subsection=false,shadow]{miniframes}
\useinnertheme{default}
\usefonttheme{serif}
\usepackage{palatino}
\usepackage{tabu}
% Links
\usepackage{hyperref}
\definecolor{links}{HTML}{003262}
\hypersetup{colorlinks,linkcolor=,urlcolor=links}

% addition of color
\usepackage{xcolor}
\definecolor{CoolBlack}{rgb}{0.0, 0.18, 0.39}
\definecolor{byellow}{rgb}{0.55037, 0.38821, 0.06142}
\definecolor{dgreen}{rgb}{0.,0.6,0.}
\definecolor{RawSienna}{cmyk}{0,0.72,1,0.45}
\definecolor{forestgreen(web)}{rgb}{0.13, 0.55, 0.13}
\definecolor{cardinal}{rgb}{0.77, 0.12, 0.23}

\setbeamerfont{title like}{shape=\scshape}
\setbeamerfont{frametitle}{shape=\scshape}

\setbeamercolor*{lower separation line head}{bg=CoolBlack} 
\setbeamercolor*{normal text}{fg=black,bg=white} 
\setbeamercolor*{alerted text}{fg=dgreen} % just testing; I think this looks better
\setbeamercolor*{example text}{fg=black} 
\setbeamercolor*{structure}{fg=black} 
 
\setbeamercolor*{palette tertiary}{fg=black,bg=black!10} 
\setbeamercolor*{palette quaternary}{fg=black,bg=black!10} 

% Margins
\usepackage{changepage}

\mode<presentation>
{
  \definecolor{berkeleyblue}{HTML}{003262}
  \definecolor{berkeleygold}{HTML}{FDB515}
  \usetheme{Boadilla}      % or try Darmstadt, Madrid, Warsaw, Boadilla...
  %\usecolortheme{dove} % or try albatross, beaver, crane, ...
  \setbeamercolor{structure}{fg=berkeleyblue,bg=berkeleygold}
  \setbeamercolor{palette primary}{bg=berkeleyblue,fg=white} % changed this
  \setbeamercolor{palette secondary}{fg=berkeleyblue,bg=berkeleygold} % changed this
  \setbeamercolor{palette tertiary}{bg=berkeleyblue,fg=white} % changed this
  \usefonttheme{structurebold}  % or try serif, structurebold, ...
  \useinnertheme{circles}
  \setbeamertemplate{navigation symbols}{}
  \setbeamertemplate{caption}[numbered]
  \usebackgroundtemplate{}
}

% Columns
\renewcommand{\(}{\begin{columns}}
\renewcommand{\)}{\end{columns}}
\newcommand{\<}[1]{\begin{column}{#1}}
\renewcommand{\>}{\end{column}}

\usepackage{cutwin}

% adding slide numbers
\addtobeamertemplate{navigation symbols}{}{%
    \usebeamerfont{footline}%
    \usebeamercolor[fg]{footline}%
    \hspace{1em}%
    \insertframenumber/\inserttotalframenumber
}

% equation stuff
\newcommand{\Macro}{\ensuremath{\Sigma}}
\newcommand{\Sn}{\ensuremath{S_N} }
\newcommand{\vOmega}{\ensuremath{\hat{\Omega}}}
\usepackage{mathrsfs}
\usepackage[mathcal]{euscript}
\usepackage{amssymb}
\usepackage{amsthm}
\usepackage{epsfig}
\usepackage{amsmath}
\newcommand{\ve}[1]{\ensuremath{\mathbf{#1}}}
\newcommand{\micro}{\ensuremath{\sigma}}
\newcommand{\detR}{\ensuremath{\Sigma}}

% title stuff for footer
\title{Approaches to HPC in NE}
\author{R.\ N.\ Slaybaugh}
\date{8 December 2014}

%%%%%%%%%%%%%%%%%%%%%%%%%%%%%%%%%%%%%%%%%%%%%%%%%%%%%%
\begin{document}

%%%%%%%%%%%%%%%%%%%%%%%%%%%%%%%%%%%%%%%%%%%%%%%%%%%%%%
\begin{frame}
\title{Advanced Approaches to High-Performance Computing in Nuclear:}
\subtitle{Applications to Non-Proliferation}
\author{\includegraphics[height=2cm]{../bk-eps-converted-to}\\R.\ N.\ Slaybaugh \\ Univ.\ of Cal.\ Berkeley}

\date{MIIT Delegation VISIT to BNRC \\ 8 December 2014}
\titlepage
\end{frame}

%------------------------------------------------------
\begin{frame}{What Are We Sovling?}

    My group studies how to solve the steady state, neutral particle Boltzmann
    transport equation more efficiently:
    %   
    \begin{align}
    [\vOmega \cdot \nabla + \Macro(\vec{r}, E)] &\psi(\vec{r}, \vOmega, E)  =  q(\vec{r}, \vOmega, E) + \nonumber\\
     &\int_0^{\infty} dE' \int_{4\pi} d\vOmega' \:\Macro_{s}(\vec{r}, E' \to E,
     \vOmega' \cdot \vOmega) \psi(\vec{r}, \vOmega', E') \nonumber
    \end{align}
    
    \begin{columns}
    \begin{column}{0.55\textwidth}     
 	   \begin{center}
 	   \begin{figure}
 	   %\includegraphics[height=2.25in,clip]{../figs/u235-xsecs}
 	   \includegraphics[height=1.5in,clip]{../figs/isfsi}
       \end{figure}
 	   \end{center}
  	\end{column}
   	%
 	\begin{column}{0.4\textwidth}
 	   \begin{center}
 	   \begin{figure}     
 	   %\includegraphics[height=1.in, width=1.25in, clip]{../figs/Fe-D2O-C}
 	   \includegraphics[height=1.5in,clip]{../figs/denovo-pwr}
 	   \end{figure}
 	   %\begin{figure} 
 	   %\includegraphics[height=1.in,clip]{../figs/Fe-D2O-C-space-energy}
       %\end{figure}
 	   \end{center}
  	\end{column}
    \end{columns}
   
\end{frame}

%------------------------------------------------------
\begin{frame}{How Do We Solve It?}
    
    \begin{itemize}
    \item \alert{Deterministic} methods require discretization of phase space
      \begin{itemize}
      \item discretize more finely to improve solution quality
      \item use advanced solvers to converge solution more quickly
      \end{itemize}
      %
      \begin{align}
      \mathbf{L} \psi &= \mathbf{MS}\phi + \mathbf{Q} \nonumber\\
      \phi &= \mathbf{D}\psi \nonumber \\
      \underbrace{(\ve{I} - \ve{DL}^{-1}\ve{MS})}_{\mathbf{A}}\phi &= q\nonumber
      \end{align}
      %        
    \item \alert{Monte Carlo} (MC) treats phase space continuously
      \begin{itemize}
      \item accuracy depends on number of particles simulated
      \item often requires variance reduction (VR)
      \end{itemize}      
    \item \alert{Hybrid} methods: create MC VR parameters using deterministic solutions
    \end{itemize}
    
\end{frame}

% ------------------------------------------------------
\begin{frame}{What Drives the Challenges and Solutions?}
    	
    \begin{columns}
    \begin{column}{0.55\textwidth}     
 	   \begin{center}
 	   \begin{figure}
 	   \includegraphics[height=2.25in,clip]{../figs/u235-xsecs}
       \end{figure}
 	   \end{center}
  	\end{column}
   	%
 	\begin{column}{0.4\textwidth}
 	   \begin{center}
 	   \begin{figure}     
 	   \includegraphics[height=1.25in, width=1.25in, clip]{../figs/Fe-D2O-C-space-energy}
 	   %Evans
 	   \end{figure}
 	   \begin{figure} 
 	   \includegraphics[height=1in,clip]{../figs/random-walk}
 	   %http://spiff.rit.edu/classes/phys440/lectures/walk/walk.html
       \end{figure}
 	   \end{center}
  	\end{column}
	\end{columns}
	
\end{frame}

%------------------------------------------------------
\begin{frame}{What Drives the Challenges and Solutions?}
    
    \begin{columns}
    \begin{column}{0.3\textwidth}
 	   \begin{center}
 	   \begin{figure}     
 	   \includegraphics[height=1.25in,clip]{../figs/Intel-Xeon-Phi-Board}
 	   \end{figure}
 	   \end{center}
    \end{column}
   	%
 	\begin{column}{0.8\textwidth}
 	   \begin{center}
 	   \begin{figure} 
 	   \includegraphics[height=1in,clip]{../figs/GPU}
       \end{figure}
 	   \end{center}
  	\end{column}
	\end{columns}
	
	\begin{center}
 	\begin{figure}
 	\includegraphics[height=0.75in,clip]{../figs/Titan}
    \end{figure}
 	\end{center}
 	
 	\textcolor{dgreen}{Architecture} influences algorithm choices and data management
	
\end{frame}

%%%%%%%%%%%%%%%%%%%%%%%%%%%%%%%%%%%%%%%%%%%%%%%%%%%%%%
\begin{frame}{WARP \cite{warp}}
	\begin{itemize}
	\item{Weaving All the Random Particles}
	%\pause
	\item{Originally developed by Dr. Ryan Bergmann; current development by Kelly Rowland}
	%\pause
	\item{3D continuous-energy Monte Carlo neutron transport code developed
	     for efficient implementation of the algorithm on GPUs}
	\item{Can calculate multiplication factors, flux tallies, fission source
	     distributions for time-independent problems}
	\item{Run in either criticality or fixed-source mode}
	\item{Simulates neutron transport in unrestricted arrangements of parallelpipeds,
	     hexagonal prisms, cylinders, spheres}
	%\pause
	\item{Now developing delta tracking \cite{woodcock}}
	\end{itemize}
\end{frame}

%------------------------------------------------------
\begin{frame}{Successful Performance With Reactors}
 
    \begin{center}
 	\begin{figure}     
 	\includegraphics[height=1.5in,clip]{../figs/assembly-fiss-6}
 	\end{figure}
 	\end{center}   
 	
 	\begin{center}
 	\begin{figure}     
 	\includegraphics[height=1.5in,clip]{../figs/assembly-spec-6}
 	\end{figure}
 	\end{center} 
 
\end{frame}

%%%%%%%%%%%%%%%%%%%%%%%%%%%%%%%%%%%%%%%%%%%%%%%%%%%%%%
\begin{frame}[fragile]
  \frametitle{Current Hybrid Methods are Insufficient}

	\begin{itemize}
	\item MC VR parameters created from adjoint deterministic flux that is a
	      function of \emph{space and energy only}
	\item Angular dependence of the importance function is not retained, otherwise
	      the map would be
	  \begin{itemize}
	  \item very large (tens or hundreds of GB)
	  \item more costly and complex to use in MC simulation
      \end{itemize}	   
	\item Drawback: within a given space/energy cell, the map provides the
	      \emph{average} importance of a particle moving in \emph{any direction}
	      through the cell -- excluding information about how particles move
	      \underline{toward} the objective
	\end{itemize}

\end{frame}

% --------------------------------------------------------------
\begin{frame}[fragile]
  \frametitle{Especially For Some Security Apps \cite{Peplow2012}}

	\begin{columns}
  	\begin{column}{0.5\textwidth}
 	 \begin{center}
 	 \begin{figure}
 	 \includegraphics[height=2in,clip]{../figs/boat-interrogation}  
 	 \caption{Spherical boat model with source on left and fissionable material at center}
 	 \end{figure}
 	 \end{center}
  	\end{column}
 	%
 	\begin{column}{0.5\textwidth}
 	 \begin{center}
 	 \begin{figure}
 	 \includegraphics[height=2in,clip]{../figs/boat-map}  
 	 \caption{Target CADIS weight window values for 14.1 MeV neutrons}
 	 \end{figure}
 	 \end{center}
  	\end{column}
	\end{columns}

\end{frame}

% --------------------------------------------------------------
\begin{frame}[allowframebreaks]{References}
	\bibliographystyle{unsrt}
	\bibliography{hpc-highlevel.bib}
\end{frame}

\end{document}
