%%%%%%%%%%%%%%%%%%%%%%%%%%%%%%%%%%%%%%%%%%%%%%%%%%%%%%%%%%%%
\documentclass[xcolor=x11names,compress]{beamer}

\definecolor{CoolBlack}{rgb}{0.0, 0.18, 0.39}
\definecolor{byellow}{rgb}{0.55037, 0.38821, 0.06142}
%% General document %%%%%%%%%%%%%%%%%%%%%%%%%%%%%%%%%%
\usepackage{graphicx}
\usepackage{tikz}
\usepackage{Tabbing}
\usetikzlibrary{decorations.fractals}
%%%%%%%%%%%%%%%%%%%%%%%%%%%%%%%%%%%%%%%%%%%%%%%%%%%%%%

%% Beamer Layout %%%%%%%%%%%%%%%%%%%%%%%%%%%%%%%%%%
\useoutertheme[subsection=false,shadow]{miniframes}
\useinnertheme{default}
\usefonttheme{serif}
\usepackage{palatino}
\usepackage{tabu}

% addition of color
\usepackage{xcolor}
\definecolor{dgreen}{rgb}{0.,0.6,0.}
\definecolor{RawSienna}{cmyk}{0,0.72,1,0.45}

\setbeamerfont{title like}{shape=\scshape}
\setbeamerfont{frametitle}{shape=\scshape}

\setbeamercolor*{lower separation line head}{bg=CoolBlack} 
\setbeamercolor*{normal text}{fg=black,bg=white} 
\setbeamercolor*{alerted text}{fg=dgreen} 
\setbeamercolor*{example text}{fg=black} 
\setbeamercolor*{structure}{fg=black} 
 
\setbeamercolor*{palette tertiary}{fg=black,bg=black!10} 
\setbeamercolor*{palette quaternary}{fg=black,bg=black!10} 

% Links
\usepackage{hyperref}
\definecolor{links}{HTML}{003262}
\hypersetup{colorlinks,linkcolor=,urlcolor=links}

% columns
\renewcommand{\(}{\begin{columns}}
\renewcommand{\)}{\end{columns}}
\newcommand{\<}[1]{\begin{column}{#1}}
\renewcommand{\>}{\end{column}}

% adding slide numbers
\addtobeamertemplate{navigation symbols}{}{%
    \usebeamerfont{footline}%
    \usebeamercolor[fg]{footline}%
    \hspace{1em}%
    \insertframenumber/\inserttotalframenumber
}

% equation stuff
\newcommand{\Macro}{\ensuremath{\Sigma}}
\newcommand{\Sn}{\ensuremath{S_N} }
\newcommand{\vOmega}{\ensuremath{\hat{\Omega}}}
\usepackage{mathrsfs}
\usepackage[mathcal]{euscript}
\usepackage{amssymb}
\usepackage{amsthm}
\usepackage{epsfig}
\usepackage{amsmath}

\newcommand{\ve}[1]{\ensuremath{\mathbf{#1}}}
\newcommand{\micro}{\ensuremath{\sigma}}
\newcommand{\detR}{\ensuremath{\Sigma}}
%%%%%%%%%%%%%%%%%%%%%%%%%%%%%%%%%%%%%%%%%%%%%%%%%%

\begin{document}

%%%%%%%%%%%%%%%%%%%%%%%%%%%%%%%%%%%%%%%%%%%%%%%%%%%%%%
%%%%%%%%%%%%%%%%%%%%%%%%%%%%%%%%%%%%%%%%%%%%%%%%%%%%%%
\begin{frame}
\title{Improved Hybrid Modeling of Used Fuel Storage Facilities}
\author{\includegraphics[height=2cm]
{../bk-eps-converted-to}\\R.\ N.\ Slaybaugh \\ Univ.\ of Cal.\ Berkeley}
\date{25 March 2015 \\ DOE MPACT Meeting}
\titlepage
\end{frame}

% --------------------------------------------------------------
\begin{frame}[fragile]{Outline}
  \frametitle{Outline}
  \begin{itemize}
    \item Hybrid methods overview
    \begin{itemize}
     	\item Motivation
		\item CADIS
		\item FW-CADIS
		\item Challenges
    \end{itemize}
	\item MC importances for problems with strong anisotropies
	\item 
  \end{itemize}

\end{frame}


% --------------------------------------------------------------
% --------------------------------------------------------------
\section{\scshape Hybrid Methods}
%\subsection{Motivation}
\begin{frame}[fragile]
  \frametitle{Solving the TE}

\begin{columns}
  \begin{column}{0.5\textwidth}
  \begin{center}
  \underline{Monte Carlo}
  \end{center}
	\begin{itemize}
	\item Solution has associated statistical error
	\item Continuous phase space: ``gold standard answers"
	\item Can take a long time
	\item Good for streaming
	\item Optically thick = slow
	\end{itemize}
  \end{column}
  \begin{column}{0.5\textwidth}
  \begin{center}
  \underline{Deterministic}
  \end{center}
	\begin{itemize}
	\item Solution equally valid everywhere
	\item Discretized phase space: drives solution quality
	\item Can be fast
	\item Streaming = ray effects
	\item Good for optically thick
	\end{itemize}
  \end{column}
\end{columns}

\end{frame}

% --------------------------------------------------------------
\begin{frame}[fragile]
  \frametitle{Acceleration}
  \begin{itemize}
  	\item To use MC in many applications, we need to \textit{accelerate} it
	\item Variance reduction is designed to improve the FOM:
  \end{itemize}
\begin{align}
\text{FOM} = \frac{1}{\text{R}^2\text{t}} \qquad & \text{R = relative error} \nonumber \\ 
& \text{t = time} \nonumber 
\end{align}
  \begin{itemize}
  	\item \underline{Idea}: can we use deterministic and Monte Carlo methods together to lessen the weaknesses of each?
  \end{itemize}
  $\rightarrow$ \textbf{Hybrid Methods}

\end{frame}


% --------------------------------------------------------------
%\subsection{CADIS}
\begin{frame}[fragile]
  \frametitle{Forward-Adjoint Relationship}
Define response with function $f(\ve{r}, E)$ in volume $V_r$ as
%
\begin{equation}
 R = \int_E \int_{V_r} f(\ve{r}, E) \phi(\ve{r}, E) dV dE 
 \label{eq:Response}
\end{equation}
%
\begin{columns}
  \begin{column}{0.5\textwidth}
	\begin{align}
  	H\phi &= q \quad \text{(forward)}\nonumber \\
  	%
  	H^{\dagger} \phi^{\dagger} &= q^{\dagger} \quad 
  	\text{(adjoint)}\nonumber
  	\end{align}
  \end{column}
  \begin{column}{0.5\textwidth}
  	\begin{align}
  	\langle H\phi, \phi^{\dagger} \rangle &= \langle H^{\dagger} \phi^{\dagger}, \phi \rangle \:, \text{and therefore} \nonumber \\
  	%
  	\langle q, \phi^{\dagger} \rangle &= \langle q^{\dagger}, \phi \rangle \nonumber
  	\end{align}
  \end{column}
\end{columns}
\vspace*{1 em}
\pause
If we let $q^{\dagger} = f(\ve{r}, E)$ then
%
\begin{equation}
 \langle q^{\dagger}, \phi \rangle = \langle f, \phi \rangle = R = \langle q, \phi^{\dagger} \rangle
 \label{eq:ResponseRedef}
\end{equation}
%
Eq.\ \eqref{eq:ResponseRedef} expresses that $\phi^{\dagger}$ represents the expected contribution of a source particle to the response given the source, $q$.

\end{frame}

% --------------------------------------------------------------
\begin{frame}[fragile]
  \frametitle{CADIS}
  
  \begin{enumerate}
  \item Define $q^{\dagger}$ as the local response of interest\\
  \item Coarse deterministic calculation to get $\phi^{\dagger}$ and $R$
  \end{enumerate}
% 
\begin{align}
  imp(\ve{r}, E) &= \frac{\phi^{\dagger}(\ve{r}, E)}{\langle q(\ve{r}, E), \phi^{\dagger}(\ve{r}, E) \rangle} = \frac{\phi^{\dagger}(\ve{r}, E)}{R} \\
  %
  \hat{q}(\ve{r}, E) &= \frac{\phi^{\dagger}(\ve{r}, E) q(\ve{r}, E)}{R} \\
  %
  w_0(\ve{r}, E) &= \frac{q(\ve{r}, E)}{\hat{q}(\ve{r}, E)} = \frac{R}{\phi^{\dagger}(\ve{r}, E)} 
  \label{eq:Importance}
\end{align}

Birth weights match weight targets, making this the \underline{C}onsistent \underline{A}djoint \underline{D}riven \underline{I}mportance \underline{S}ampling \underline{M}ethod

\end{frame}

% --------------------------------------------------------------
%\subsection{FW-CADIS}
\begin{frame}[fragile]
  \frametitle{FW-CADIS \cite{Wagner2007}}

\begin{itemize}
\item We often what to optimize solutions in all of phase space\\
\item In this case the adjoint source needs to be a global forward solution: \underline{F}orward \underline{W}eighted-CADIS
\end{itemize}
%
\begin{columns}
  \begin{column}{0.5\textwidth}
  \begin{center}
  \textcolor{byellow}{To Optimize}
  \end{center}
	\begin{align}
  	&\phi(\ve{r}, E)\nonumber \\
  	%
  	\int&\phi(\ve{r}, E)\sigma_d(\ve{r}, E)\nonumber
  	\end{align}
  \end{column}
  %
  \begin{column}{0.5\textwidth}
  \begin{center}
  \textcolor{byellow}{Adjoint Source}
  \end{center}
  	\begin{align}
  	f(\ve{r}, E) &= \frac{1}{\phi(\ve{r}, E)}\nonumber \\
  	%
  	f(\ve{r}, E) &= \frac{\sigma_d(\ve{r}, E)}{\int\phi(\ve{r}, E)\sigma_d(\ve{r}, E)} \nonumber
  	\end{align}
  \end{column}
\end{columns}
\vspace*{1 em}
\pause
For example
%
\begin{equation}
 R = \int_E \int_{V_f} f(\ve{r}, E) \phi(\ve{r}, E) dV dE = \int_E \int_{V} \frac{1}{\phi(\ve{r}, E)} \phi(\ve{r}, E) dV dE \approx 1 \nonumber
\end{equation}

\end{frame}

% --------------------------------------------------------------
%\subsection{Challenges}
\begin{frame}[fragile]
  \frametitle{Challenges}

	FW-CADIS works well for \textbf{most} deep penetration
	shielding problems...
	%
	\begin{columns}
  	\begin{column}{0.5\textwidth}
  	\begin{figure}
  	\begin{center}
  		\includegraphics[height=2in,clip]{../figs/dlvn}
		\caption{Dog Legged Void Neutron shielding benchmark}
	\end{center}
  	\end{figure}
  	\end{column}
 	%
 	\begin{column}{0.5\textwidth}
 	\begin{figure}
  	\begin{center}
  		\includegraphics[height=2in,clip]{../figs/dlvn-lowVR}
  		\caption{MC 95\% CI RE using FW-CADIS, DLVN \cite{Slaybaugh2013}}
  	\end{center}
  	\end{figure}
  	\end{column}
	\end{columns}
  
\end{frame}

% --------------------------------------------------------------
\begin{frame}[fragile]
  \frametitle{Challenges}

	...but not all of them
	%
	\begin{columns}
  	\begin{column}{0.5\textwidth}
  	\begin{center}
		\begin{itemize}
		\item FW-CADIS only includes space and energy, 
			\textit{not angle}
		\item One pathological case:
		\begin{itemize}
		\item Energy self-shielding +
		\item Spatial self-shielding
		\end{itemize}
		\item High relative error through location of interest
		\item \textbf{Need} new methods based on FW-CADIS
		\item E.g., Resonance Factor method (uses different cross section processing)
		\end{itemize}
	\end{center}
  	\end{column}
 	%
 	\begin{column}{0.5\textwidth}
  	\begin{center}
  	\begin{figure}
  		\includegraphics[height=2in,clip]{../figs/plate-badVR}
  		\caption{MC 95\% CI RE using FW-CADIS, plate \cite{Wilson2015}}
  	\end{figure}
  	\end{center}
  	\end{column}
	\end{columns}
  
\end{frame}


% --------------------------------------------------------------
% --------------------------------------------------------------
\section{Strong Anisotropies}
\begin{frame}[fragile]
  \frametitle{Anisotropy: a computational challenge}

	\begin{columns}
  	\begin{column}{0.5\textwidth}
	\begin{itemize}
	\item Many important nuclear applications have strong anisotropies
	 \begin{itemize}
	 \item Used fuel casks
	 \item Reprocessing facilities
	 \item Reactor facilities
	 \item Active interrogation 
	 \end{itemize}
	\pause
	\item These are difficult to capture with current tools:
	 \begin{itemize}
	 \item Ray effects with deterministic
	 \item Too slow with analog MC
	 \item Insufficient acceleration of MC with hybrid
	 \end{itemize}
	\end{itemize}
  	\end{column}
 	%
 	\begin{column}{0.5\textwidth}
 	 \begin{center}
 	 \begin{figure}
 	 \includegraphics[height=2in,clip]{../figs/pwr}  
 	 \caption{PWR, 1 CPU-month, FW-CADIS  for mesh-tally (500K cells)}
 	 \end{figure}
 	 \end{center}

  	\end{column}
	\end{columns}

\end{frame}


% --------------------------------------------------------------
\begin{frame}[fragile]
  \frametitle{Current hybrid methods are insufficient}

	\begin{itemize}
	\item MC VR parameters created from adjoint deterministic flux that is a function of space and energy only \vspace*{1 em}
	\item Angular dependence of the importance function is not retained, otherwise the map would be very large (tens or hundreds of GB) and more costly and complex to use in the Monte Carlo simulation \vspace*{1 em}
	\item Drawback: within a given space/energy cell, the map provides the average importance of a particle moving in any direction through the cell -- excluding information about how particles move toward the objective
	\end{itemize}

\end{frame}

% --------------------------------------------------------------
\begin{frame}[fragile]
  \frametitle{Current hybrid methods are insufficient}

	\begin{columns}
  	\begin{column}{0.5\textwidth}
 	 \begin{center}
 	 \begin{figure}
 	 \includegraphics[height=2in,clip]{../figs/boat-interrogation}  
 	 \caption{Spherical boat model with source on left and fissionable material at center}
 	 \end{figure}
 	 \end{center}
  	\end{column}
 	%
 	\begin{column}{0.5\textwidth}
 	 \begin{center}
 	 \begin{figure}
 	 \includegraphics[height=2in,clip]{../figs/boat-map}  
 	 \caption{Target weight window values for 14.1 MeV neutrons}
 	 \end{figure}
 	 \end{center}
  	\end{column}
	\end{columns}

\end{frame}

% --------------------------------------------------------------
\begin{frame}[fragile]
  \frametitle{Many Attempts at Resolution; Mixed Success}

	\begin{itemize}
	\item Automatic WW generator
	\item AVATAR
	\item LIFT
	\item Cooper and Larsen's global weight windows
	\item CADIS/FW-CADIS
	\item Resonance Factor method
	\end{itemize}

\end{frame}

% --------------------------------------------------------------
\begin{frame}[fragile]
  \frametitle{Better hybrid methods are needed}

We may be able to use angular information to improve performance; two ideas: \vspace*{1 em}

	\begin{enumerate}
	\item More effectively capture angles in scalar flux creation
		\begin{equation}
		\phi^{\dagger}(\ve{r},E) = \frac{\int \psi(\vOmega, \ve{r},E) \psi^{\dagger}(\vOmega, \ve{r},E) d\vOmega}{\int \psi(\vOmega, \ve{r},E)  d\vOmega}
		\end{equation}

	\item Use a formulation that is more flexible in the ways that 
	it handles quadrature: new LDO equations \cite{Ahrens2014}
	\end{enumerate}

\end{frame}

% --------------------------------------------------------------
\begin{frame}[fragile]
  \frametitle{Lagrange Discrete Ordinate Equations}

    Cory Ahrens from Colorado School of Mines
	\begin{itemize}
	\item Re-derivation of $S_N$ with an interpolatory quadrature framework
	\item Allows evaluation at directions not on quadrature set
	\item No need to store spherical harmonic moments
	\item May be useful for more accurately capturing strong anisotropies
	\end{itemize}
	
	Use as standalone or in FW-CADIS?

\end{frame}

% --------------------------------------------------------------
% --------------------------------------------------------------
\begin{frame}[fragile]
  \frametitle{Better hybrid methods are needed}

  	\begin{itemize}
    \item The space- and energy-dependent importance map will be normalized and 
    source biasing parameters will be generated in a manner similar to the current
     implementation of hybrid methods \vspace*{1 em}
	\item Immediately useful; widely applicable \vspace*{1 em}
	\item We will study both strategies and characterize the impact
	\end{itemize}
	
\end{frame}


% --------------------------------------------------------------
% --------------------------------------------------------------
\section*{}
\begin{frame}[fragile]
  \frametitle{Questions?}
  \begin{center}
  \includegraphics[height=3in,clip]{../questions-comic}  
  \end{center}
  
\end{frame}

% --------------------------------------------------------------
\begin{frame}[allowframebreaks]{References}
	\bibliographystyle{unsrt}
	\bibliography{2015-03-mpact}
\end{frame}

\end{document}
