% Talk for PSI, July 2015.

\documentclass[12pt]{article}
\usepackage{fullpage}
\usepackage{paralist}
\usepackage{hyperref}

\title{Hybrid Transport Methods for Shielding Challenges}
\author{R.\ N.\ Slaybaugh \\ Univ.\ of Cal.\ Berkeley}
\date{16 July 2015}

\begin{document}

\maketitle

\section*{Abstract}

Continued demand for accurate and computationally efficient transport methods to solve optically thick, fixed-source transport problems has inspired research on variance reduction techniques for Monte Carlo (MC). Methods that use deterministic results to create variance reduction maps for MC constitute a dominant branch of this research, with FW-CADIS being a particularly successful example. However, there are important physical configurations in which this method does not perform adequately including (1) locations in which energy and spatial self-shielding are combined, such as thin plates embedded in concrete and (2) problems containing strong anisotropies, such as streaming locations and active interrogation problems. In these cases the deterministic flux cannot appropriately capture transport behavior, and the associated variance reduction parameters result in high variance. 

This presentation will focus on a new method that improves performance in transport calculations containing regions of combined space and energy self-shielding without significant impact on the solution quality in other parts of the problem. It is clear that this new method dramatically improves performance in terms of lowering the maximum relative error and reducing the compute time. The new method will be followed by discussion of research on methods for dealing with the anisotropic challenge.

\section*{Biography}
Rachel Slaybaugh is an Assistant Professor of Nuclear Engineering at the University of California, Berkeley. At Berkeley, Prof.\ Slaybaugh's research program is based in computational science and applied to existing and advanced nuclear reactors, nuclear non-proliferation and security, and shielding applications. She received a BS in Nuclear Engineering from Penn State in 2006 where she served as a licensed nuclear reactor operator. Dr.\ Slaybaugh went on to the University of Wisconsin -- Madison to earn an MS in 2008 and a PhD in 2011 in Nuclear Engineering and Engineering Physics along with a certificate in Energy Analysis and Policy. For her PhD she researched acceleration methods for massively parallel deterministic neutron transport codes. Dr.\ Slaybaugh then worked with hybrid (deterministic-Monte Carlo) methods for shielding applications at Bettis Laboratory while teaching at the University of Pittsburgh as an adjunct faculty member. Throughout her career, Dr.\ Slaybaugh has been engaged in Software Carpentry education and training; she also contributes to the open source project PyNE (\href{http://pyne.io}{http://pyne.io}). Prof.\ Slaybaugh was awarded the 2014 American Nuclear Society Young Member Excellence Award.

\end{document}
