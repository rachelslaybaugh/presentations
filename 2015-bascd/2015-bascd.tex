\documentclass[final]{beamer}

\mode<presentation>
{
  \definecolor{berkeleyblue}{HTML}{003A70}
  \definecolor{berkeleygold}{HTML}{F2A900}
  %\definecolor{berkeleyblue}{HTML}{003262}
  %\definecolor{berkeleygold}{HTML}{FDB515}
%http://www.berkeley.edu/brand/img/downloads/UCB%20Brand%20Guidelines_FINAL_small.pdf
  \usetheme{I6dv}      % or try Darmstadt, Madrid, Warsaw, ...
  %\usecolortheme{dove} % or try albatross, beaver, crane, ...
  \setbeamercolor{structure}{fg=berkeleygold,bg=berkeleyblue}
  \setbeamercolor{palette primary}{fg=berkeleyblue,bg=berkeleygold} % changed this
  \setbeamercolor{palette secondary}{bg=berkeleyblue,fg=white} % changed this
  \setbeamercolor{palette tertiary}{bg=berkeleyblue,fg=white} % changed this
\setbeamercolor{bibliography item}{bg=white,fg=black}
\setbeamercolor{bibliography entry author}{bg=white,fg=black}
\setbeamercolor{bibliography entry journal}{bg=white,fg=black}
\setbeamercolor{bibliography entry note}{bg=white,fg=black}
  \usefonttheme{structurebold}  % or try serif, structurebold, ...
  \useinnertheme{circles}
  \setbeamertemplate{navigation symbols}{}
  \setbeamertemplate{footline}[title]{}
  \setbeamertemplate{footline}[author]{}
  \usebackgroundtemplate{
%  \tikz[overlay,remember picture] %Can insert a Berkeley watermark
%  \node[opacity=1, at=(current page.south west),anchor=south west, inner sep=2pt] {
%    \includegraphics[height=.5in,width=.5in]{img/ucseal_540_139.eps}};
}
}
\definecolor{dgreen}{rgb}{0.,0.6,0.}
\definecolor{rawsienna}{cmyk}{0,0.72,1,0.45}
%\renewcommand*{\bibfont}{\scriptsize}
\usepackage[sorting=none]{biblatex} 
  \usepackage{ragged2e}
  \usepackage{type1cm}
  \usepackage{calc} 
  \usepackage{times}
  \usepackage{amsmath,amsthm, amssymb, latexsym}
%  \boldmath
  \usepackage[english]{babel}
  \usepackage[latin1]{inputenc}
  \usepackage[orientation=landscape,size=a0,scale=1.4,debug]{beamerposter}
  \graphicspath{{img/}}
  %\title[Fancy Posters]{Making Really Fancy Posters with \LaTeX}
  %\author[Dreuw \& Deselaers]{Philippe Dreuw and Thomas Deselaers}
  %\institute[University of California, Berkeley]{Department of Nuclear Engineering}
  \newcommand{\footlinetext}{}
  \usepackage{biblatex}

% Transport equation stuff
\newcommand{\Sn}{\ensuremath{S_N}}
\newcommand{\Macro}{\ensuremath{\Sigma}}
\newcommand{\vOmega}{\ensuremath{\hat{\Omega}}}
\newcommand{\Ye}[2]{\ensuremath{Y^e_{#1}(\vOmega_#2)}}
\newcommand{\Yo}[2]{\ensuremath{Y^o_{#1}(\vOmega_#2)}}
\newcommand{\ve}[1]{\ensuremath{\mathbf{#1}}}

\setbeamertemplate{bibliography item}[text]
\addbibresource{2015-bascd.bib}
\renewcommand*{\bibfont}{\footnotesize}


\title[RQI with MGE]{RQI with a Multigrid in Energy Preconditioner for Massively Parallel Neutron Transport}
\author[R.N.\ Slaybaugh et al.]{R.N.\ Slaybaugh$^{1}$, T.M.\ Evans$^{2}$, G.G. Davidson$^{2}$, S.P.\ Hamilton$^{2}$}
  \institute[UC Berkeley]{$^{1}$Department of Nuclear Engineering, University of California, Berkeley\\
  $^{2}$Radiation Transport Group, Oak Ridge National Laboratory}
  \date{December 11, 2015}

\begin{document}
	\begin{frame}{}
%    \maketitle

%%%%%%%%%%%%%%
% FIRST COLUMN
%%%%%%%%%%%%%%

  		\begin{columns}[t]
    		\begin{column}{.3\linewidth}
    			\vfill
    			\begin{block}{\large Motivation}
      			\begin{itemize}
			\item{Improving nuclear reactor design, shielding, and security}
			\item{3D deterministic calculations need fine resolution in space, 
			      energy, and angle}
			\item{Need to use supercomputing architectures effectively}
			\item{Three complimentary methods: multigroup (MG) block Krylov, 
			      Rayleigh quotient iteration (RQI), multigrid in energy (MGE)
			      preconditioner}
			\end{itemize}
    			\end{block}
    	\vfill
    			\begin{block}{\large Boltzmann Transport Equation}
			\begin{itemize}
	 		\item{Solve for $\psi$, the angular neutron flux (n/cm$^2$-s-steradian)}
	 		\begin{equation}
	 		  \ve{L}\psi = \ve{MS}\phi + \frac{1}{k}\ve{M}\chi \ve{f}^{T}\phi	                
	 		  \label{eq:eigenvalue}
	 		\end{equation}	 		
	 		\item{Collocation method in angle: $\phi = \mathbf{D} \psi$ (n/cm$^2$-s)}
	 		\item{$\ve{L}$ is the $\vOmega \cdot \nabla + \Macro_t$ operator; $\ve{M}$
	 		 projects $\phi$ onto discrete angles; $\ve{S}$
	 		 is the scattering matrix; $\ve{f}$ contains the fission source, 
	 		 $\nu \Macro_{f}$; and $k$ is the asymptotic ratio of the number of 
	 		 neutrons in one generation to the next}
	 		\item{Rearrange Eqn.~\eqref{eq:eigenvalue}; define $\ve{T} = \ve{DL}^{-1}$ 
	 		 and $\ve{F} = \chi \ve{f}^{T}$ \cite{denovo}}:
            \begin{equation}
              (\ve{I} - \ve{TMS})\phi = \frac{1}{k} \ve{TMF} \phi
              \label{eq:OperatorEvalForm}
            \end{equation}
            \item{Get series of ``within-group'' eqns., each only a function of space
             and angle}
            \item{Groups can be coupled by scattering from low to higher energy
             $\rightarrow$ iterative ``multigroup'' solves over the coupled portion needed}
			\end{itemize}
    			\end{block}
    	\vfill
        	\begin{block}{\large Multigroup Block Krylov \cite{Davidson2013}}		
		\begin{itemize}
		\item{Groups historically solved in series with Gauss Seidel} 
			\begin{figure}[h!]
			\centering
	      		\includegraphics{block-krylov.png}
			%\caption{MG Krylov group blocking}
			\end{figure}
		\item{In Denovo~\cite{denovo} solve downscatter groups individually}
		\item{Solve upscatter block together with multi-group-sized iteration vector 
		      in GMRES}
		\item{Enables parallelization in energy (``multisets")}
		\end{itemize}
        	\end{block}
    	\end{column}

%%%%%%%%%%%%%%%
% SECOND COLUMN
%%%%%%%%%%%%%%%

    \begin{column}{.3\linewidth}
    \begin{block}{Rayleigh Quotient Iteration \cite{Slaybaugh2012}}
	\begin{itemize}
	\item{Eigenvalues historically found with power iteration (PI)}
	\begin{equation}
	(\ve{I} - \ve{TMS})^{-1} \ve{TMF} \phi = k\phi
	\end{equation}
	\vspace*{-1em}
		\begin{itemize}
		\item{Error reduced as $\lambda_{2}/\lambda_{1}$; slow in many nuclear cases}
		\end{itemize}
		\vspace*{0.2 em}
	\item{Shifted inverse iteration (PI on $(\ve{A} - \mu \ve{I})^{-1}$) gives same eigenvectors; $\sigma([\ve{A} - \mu \ve{I}]^{-1}) = \{1/(\lambda - \mu):\lambda \in \sigma(\ve{A})\}$ }
		\vspace*{0.1 em}
		\begin{itemize}
		\item{Error now reduced as $(\lambda_{1}-\mu)/(\lambda_{2}-\mu)$}
		\end{itemize}		
		\vspace*{0.1 em}
	\item{RQI uses Rayleigh quotient as an \textit{optimal} shift}
		\begin{equation}
          \rho = \frac{\phi^T (\ve{I} - \ve{TMS})\phi}{\phi^T \ve{TMF} \phi} \approx \frac{1}{k}
		\end{equation}
		\item{RQI in Denovo: subtract $\rho \ve{TMF}$ from both sides of 
		 Eqn.\ \eqref{eq:OperatorEvalForm}}
        \begin{equation}
          (\ve{I} - \ve{TM}\ve{\tilde{S}})\phi =( \gamma - \rho) \ve{TMF} \phi 
          \label{eq:OperatorShiftedEval} 
        \end{equation}
		\item{$\ve{\tilde{S}} \equiv \ve{S} + \rho\ve{F}$ is energy-block dense;
		 need MG Krylov}
		\item{The RQ shift can cause ill conditioning; difficult to converge
		 eigenvector iterations with Krylov}
	\end{itemize}
            \end{block}
		\vfill
        	\begin{block}{\large Multigrid in Energy \cite{Slaybaugh2013}}
        \begin{itemize}
			\item{Multigrid in energy as a right preconditioner}
			\item{Relaxation method is weighted Richardson}
			\begin{equation}
              \phi^{m} = \bigr(\ve{I} + \omega(\ve{TMS} - \ve{I})\bigl)\phi^{m-1}
               + \omega b^{m-1}
            \label{eq:relax}
            \end{equation} 
			\item{Parallelization: each energy set restricts, prolongs, and relaxes only
			 its own groups; does not need to communicate with other energy sets
			 beyond for upscattering in the solver}
			\item{Can use reduced angular quadrature in the preconditioner}
			\end{itemize}
        	\end{block}
			\vfill
		\begin{block}{\large Results\textemdash Past Work 
		              \cite{Slaybaugh2012}, \cite{Slaybaugh2013}}
\begin{columns}
	\column{0.45\linewidth}
	\begin{figure}[h!]
	\includegraphics[width=5in]{../figs/denovo-pwr.png}
	\caption{\footnotesize Full PWR; $S_{12}$ (168 angle sets), 233,858,800 
	         cells, 44 groups: 1.73 trillion unknowns}
	\end{figure}
	\column{0.55\linewidth}
	\begin{itemize}
	\item{RQI could solve easy problems without preconditioning, but not hard ones}
	\item{Using a reduced quadrature set in MGE is highly valuable}
	\item{No need to have MGE restrict to 1 group; a few grids are sufficient}
	\item{MGE makes PI slower}
	\item{MGE scales very well in energy}
	\end{itemize}
\end{columns}
\vspace*{0.5 em}
	\begin{itemize}
	\item{\textit{Will preconditioning with MGE facilitate the use of RQI?}}
	\item{\textit{Will the combination be better for some problems?}}
	\end{itemize}
		\end{block}
	\vfill
      \end{column}

%%%%%%%%%%%%%%
% THIRD COLUMN
%%%%%%%%%%%%%%

\begin{column}{.3\linewidth}
   \begin{block}{Results\textemdash RQI + MGE}
	 \begin{itemize}
	 \item{2D C5G7 benchmark: RQI converged with enough preconditioning, but was slow}
	 \item{3D C5G7 benchmark: preconditioned RQI faster than preconditioned PI; 
	  still slower than unpreconditioned PI}
	 \end{itemize}
%     \vspace*{-0.5 em}
%     \begin{table}[h]
%     \caption{3-D C5G7 benchmark: RQI and PI comparison}
%     \begin{tabular}{ c  c  c  c  c  c  c }
%     \hline
%     Sovler & Preconditioner & N Krylov & N Eigen & Time (s) \\\hline 
%     PI  & \textcolor{dgreen}{none}       & 3,129  & 32       & \textcolor{dgreen}{4.46 $\times$ 10$^{3}$} \\
%     RQI & w1.3 r2 v2 & 302    & 19       & 2.32 $\times$ 10$^{4}$ \\
%     PI  & w1.3 r2 v2 & 288    & 32       & 2.84 $\times$ 10$^{4}$ \\
%     %\hline
%     RQI & w1   r3 v3 & 103    & 9$^{*}$  & 3.02 $\times$ 10$^{4}$ \\
%     PI  & w1   r3 v3 & 126    & 14$^{*}$ & 4.04 $\times$ 10$^{4}$ \\
%     %\hline
%     RQI & w1.5 r3 v3 & 187    & 19       & 3.24 $\times$ 10$^{4}$ \\
%     PI  & w1.5 r3 v3 & 192    & 32       & 3.73 $\times$ 10$^{4}$ \\
%    \hline 
%    \end{tabular}\\
%    $^{*}$different tolerances
%    \label{table:PI RQI}
%	\end{table}	 
\vspace*{-0.3 em}
	 \begin{table}[h!]
  \caption{Full PWR: PI and RQI with and without preconditioning}
    \begin{tabular}{ c  c  c  c  c  c  c }
      \hline
      Method & Preconditioner & N Krylov & N Eigen & Time (m) \\
      \hline
      %RQI & none   & n/a & n/a   & n/a                                & n/a$^*$ \\
      %RQI & w1r1v1 & 3   & 45    & 1.252 $\pm$ 1.42 $\times$ 10$^{-2}$ & 120$^*$ \\
      RQI & \textcolor{dgreen}{w1r2v2} & 70    & 5   & \textcolor{dgreen}{54.8} \\
      RQI & w1r3v3 & 76    & 6   & 330.4$^{\dag}$ \\
      PI  & \textcolor{rawsienna}{none}   & 5,602 & 149 & \textcolor{rawsienna}{612.2} \\
      PI  & w1r2v2 & 946   & 86  & 720$^*$ \\
      PI  & w1r3v3 & 111   & 11  & 480$^{*,\dag}$ \\
      %\hline
      %PI  & none   & 24  & 901  & 1.272 $\pm$ 1.01e-4 & 92.1$^+$ \\
      %PI  & w1r2v2 & 28  & 312  & 1.273 $\pm$ 8.24e-5 & 250.9$^+$ \\
      \hline
    \end{tabular}\\
    \footnotesize{$^{*}$did not converge within walltime limit (8 or 12 hours)\\
    $^{\dag}$used $S_{12}$ in MGE preconditioner; diff.\ tolerances and decomp.}\\
  \label{tab:PWR all}
\end{table}
%
	 \begin{itemize}
	 \item{Full PWR: \textit{preconditioned RQI was fastest!}}
	 \item{Full PWR: RQI + MGE strong-scales well}
	 \end{itemize}
\vspace*{-0.3 em}
%	\begin{figure}[h!]
%	\includegraphics[width=6in]{strong-scaling.png}
%	\caption{\footnotesize Full PWR, RQI+MGE strong scaling}
%	\end{figure}
\begin{table}[h!]
  \caption{Full PWR: RQI strong scaling with $w1r2v2$ preconditioning}
  \begin{tabular}{ c  c  c  c  c  c  c }
     \hline
      Sets & Domains & Time (m) & $\text{t}_{\text{perfect}}$ & Efficiency \\
      \hline
      1   & 12,544 & 407.8 & 407.8 & 1.000 \\
      4   & 50,176 & 123.4 & 102.0 & 0.826\\
      11  & 137,984 & 54.8 & 37.1  & 0.676\\
      22  & 275,968 & 39.6 & 18.5  & 0.468\\
      \hline
  \end{tabular}
  \label{tab:PWR rqi strong scaling}
\end{table}
\vspace*{0.3 em}
          	\begin{itemize}
          	\item{RQI + MGE performed well on a large machine for a challenging problem}
          	\end{itemize}
		\end{block}
%			\vfill
%        	\begin{block}{\large Conclusions}
%          	\begin{itemize}
%          	\item{RQI is desirable for loosely coupled systems but need preconditioning}
%          	\item{Multigrid in Energy preconditioner can converge flux so RQI can
%          	 converge}
%          	\item{Together, they performed well on a large machine for a challenging problems}
%          	\end{itemize}
%        	\end{block}
			\vfill
        	\begin{block}{\large References}
			\printbibliography
        	\end{block}
			\vfill
        	\begin{block}{\large Acknowledgements}
		\justifying
        \small{This research used resources of the OLCF at the ORNL, which is supported 
        by the Office of Science of the U.S.\ Department of Energy under Contract 
        No.\ DE-AC05-00OR22725. Additional support came from the Rickover Fellowship
        Program in Nuclear Engineering sponsored by Naval Reactors Division of the 
        U.S.\ Department of Energy.}
        	\end{block}
      \end{column}
    \end{columns}
  \end{frame}
\end{document}

%    			\begin{block}{\large Fontsizes}
%      				\centering
%      				{\tiny tiny}\par
%      				{\scriptsize scriptsize}\par
%      				{\footnotesize footnotesize}\par
%      				{\normalsize normalsize}\par
%      				{\large large}\par
%      				{\Large Large}\par
%      				{\LARGE LARGE}\par
%      				{\veryHuge VeryHuge}\par
%      				{\VeryHuge VeryHuge}\par
%      				{\VERYHuge VERYHuge}\par
%    			\end{block}

%%%%%%%%%%%%%%%%%%%%%%%%%%%%%%%%%%%%%%%%%%%%%%%%%%%%%%%%%%%%%%%%%%%%%%%%%%%%%%%%%%%%%%%%%%%%%%%%%%%%
%%% Local Variables: 
%%% mode: latex
%%% TeX-PDF-mode: t
